\documentclass[11pt]{article}
\usepackage[utf8]{inputenc}
\usepackage[english]{babel}
\usepackage[a4paper, portrait, margin=1in]{geometry}
\usepackage{fancyhdr,lipsum}
\usepackage{graphicx}
\usepackage{xcolor}
\usepackage{enumitem}
\usepackage{hyperref}
\usepackage[font=footnotesize,labelfont=bf]{caption}
\usepackage{amsmath}

\pagestyle{fancy} 
\fancyhf{} 
\fancyhead[L]{Andrea Giarduz}
\fancyhead[R]{DIMA}
\fancyhead[C]{Veronica Grosso}
\fancyfoot[C]{\thepage}

\title{DIMA -- Idea Proposal}
\author{Andrea Giarduz - Veronica Grosso}

\begin{document}

\begin{center}
    \LARGE{\textbf{Mobile Application Idea Proposal}}
\end{center}

\section{Idea: subscription manager}
How many times do we forget how many subscriptions have we activated, when do they renew and how much do we spend on them? Automatic payments and shared memberships can hide and be forgotten by their owners, resulting in money loss and useless subscriptions.
\\

Our idea is to provide a mobile application that keeps track of all the memberships that the user activated. We want to help people be aware of how much money they are spending and in which subscriptions, reminding them of each renewal date.

In this way, the user is always in control of their finances, without forgetting where their money is going, when the payment is due and whether the specific subscription is worth keeping. \\

The user should then be able to add, modify and delete their subscriptions on the app. The details needed should be the name of the membership, the renewal date, the amount of money due and, optionally, which card is being currently used to pay for it and if they are sharing the subscription with any friend. In this last case, the user knows they owe money to a friend, even if the subscription may not be in their name.

A system of notifications and reminders can be set to help the user be aware of what they should manually pay and what is automatically renewed.

\section{Technologies}
The technologies that we would like to employ to develop this app are \textbf{React Native} for the front end and \textbf{Firebase} for the back end to manage the application and the database.

The tests will be handled by \textbf{Jest} for the front end.


\section{User Interface}
Ideally, the possible screens that we want to include in our application are:
\begin{itemize}
    \item the login/sign up page;
    \item the list of active subscriptions (it is possible to edit and deactivate them);
    \item the add subscription page, where the user inserts all the necessary details about the new membership;
    \item the list of friends with whom we are sharing the subscriptions;
    \item the settings page (with profile details and the customization of the reminders/notification system).
\end{itemize}

\section{Team}
The group consists in two team members:
\begin{itemize}
    \item[-] Andrea Giarduz, 10631948;
    \item[-] Veronica Grosso, 10609285.
\end{itemize}

\end{document}